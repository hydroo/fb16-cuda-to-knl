\chapter{Das DDAFA-Rekonstruktionsprogramm}

Das Programm DDAFA (Dresden Accelerated FDK Algorithm) befindet sich seit November 2015 am Helmholtz-Zentrum Dresden-Rossendorf in der Entwicklung.
Es löste dort ein internes Programm ab, das unter anderem auf MATLAB basierte, mit einem alten Borland C++ Compiler für Windows kompiliert wurde und
ausschließlich auf der x86-Architektur (32-bit) lauffähig war. Die Entwicklung von DDAFA verfolgt daher drei primäre Ziele:

\begin{itemize}
\item hohe Performanz, sicherzustellen durch Ausnutzung von Parallelität,
\item Portabilität auf andere Architekturen und Betriebssysteme, wobei das primäre Ziel Linux auf der x86-Architektur (64-bit) ist sowie
\item einfache Wartbarkeit.
\end{itemize}

Die Erfüllung des ersten und vordringlichsten Ziels wird durch den Einsatz des NVIDIA\textregistered\ CUDA\textregistered\-Programmiermodells angestrebt,
während sich die anderen beiden durch eine stark am C++11-Standard orientierte Programmierung sowie eine schlanke, modulare Programmstruktur ergeben.

\section{Programmstruktur}

Das zu lösende Problem lässt sich in die folgenden Teilprobleme zergliedern:

\begin{enumerate}
\item Wichtung der Projektionen
\item Filterung der Projektionen
\item Rückprojektion 
\end{enumerate}

Diese werden in den folgenden Abschnitten näher beschrieben.

\subsection{Wichtung der Projektionen}

Vor der gefilterten Rückprojektion müssen die Projektionen gewichtet werden. Dies geschieht für jedes Pixel einzeln nach der folgenden Formel:

\fxnote{Formel und Begründung einfügen}

\subsection{Filterung der Projektionen}

Die gewichteten Projektionen müssen vor der Rückprojektion gefiltert werden. Zu diesem Zweck werden sie zunächst Fourier-transformiert. Im komplexen
Raum wird der (ebenfalls transformierte) Filter auf das transformierte Bild angewendet und das Ergebnis anschließend zurück transformiert.

\fxnote{Filterbeschreibung}

\subsection{Rückprojektion mit der FDK-Methode}

Die gefilterten Projektionen werden im letzten Schritt zurückprojiziert. Für diesen Zweck wird der von Feldkamp, Davis und Kress (FDK) vorgestellte
Algorithmus verwendet. Jedes Voxel im Zielvolumen kann dabei unabhängig von den anderen Voxeln berechnet werden.

\fxfatal[draft]{Längere Beschreibung und Quellen}
